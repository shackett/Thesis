\section{Statistical Properties of Mass Spectrometry Data}

\subsection{Introduction}

As biologists continue to search for a deeper understanding of systems, there is an increasing focus on analyzing the proteins and metabolites that are the primary effectors of physiology and metabolism. Improved analytical techniques for measuring these species, particularly liquid/gas chromatography (GC/LC) coupled mass spectrometry (MS) has largely facilitated this interest. While there is great promise for the future of mass spectrometry, the scientific community is far from reaching a consensus on how to best analyze mass spectrometry data.  Issues of inconsistent treatment include which distribution mass spectrometry data supposedly follows, sources of variation and in proteomics, how peptide-level information should be aggregated to estimate protein abundance. More troubling than these general inconsistencies is the absence of side-by-side comparisons of alternative methods in addition to a general lack of evaluation of parametric model assumptions. 

Modern approaches to quantitatively model protein, peptide, or metabolite relative abundance generally either treat mass spectrometry data as approximately log-normal \cite{Cox:2008ir, Oberg:2012bm,Navarro:2014ke, Breunig:2014bu}, or analyze protein spectral counts as a quasipoisson trait \cite{Li:2010bj}. Across proteomics approaches that treat peptides as approximately log-normal, variability at two levels has been considered: peptide-level variance \cite{Navarro:2014ke} and signal-intensity dependent variance \cite{Zhu:2011jr,Oberg:2012bm,Navarro:2014ke}. While one or both of these sources of variation is frequently ignored \cite{Oberg:2012bm}, there is an increasing effort to account for the combined influence of multiple variance components \cite{Navarro:2014ke}. Such approaches are likely important; however to date, such studies have not clarified if under the assumed variance model, approximate log-normality is justified. Additionally, it is unclear whether peptide-level variance or signal-intensity dependent variance is the largest contributor to uncertainty in mass spectrometry data.

To benchmark alternative statistical approaches for analyzing mass spectrometry data, three moderately sized mass spectrometry experiments (\hyperref[ch-quant_anal:mass_spec_design]{Table \ref{ch-quant_anal:mass_spec_design}}) were investigated using a standardized workflow (\hyperref[ch-quant_anal:MSworkflow_p1]{Figure \ref{ch-quant_anal:MSworkflow_p1}}), which allowed for direct model comparison and assumption evaluation.  The three \textit{S. cerevisiae} datasets used for this analysis were: (1) Boer et al. 2010, which examines the role of variable chemostat nutrient environment in driving metabolomic variation; (2) Hackett et al. 2015, further described in \hyperref[ch:simmer]{Chapter \ref{ch:simmer}}, which similar to Boer et al. 2010, investigates how protein levels are affected by nutrient environment; and (3) Foss et al. 2007, which explores how protein levels are impacted by segregating variation between two parental strains. These datasets differ in terms of whether metabolites or peptides are analyzed and vary greatly in terms of both number of features and samples analyzed. Collectively, then, such investigations provide a good survey of the variable needs in mass spectrometry data analysis. The aim of each dataset is to estimate the relative abundance of features for each categorical main effect, while also minimizing the added variation that random block-specific effects introduce into the data (\hyperref[ch-quant_anal:mass_spec_design]{Table \ref{ch-quant_anal:mass_spec_design}}) \cite{Bates:2013vy}.

\begin{table}[h!]
\begin{center}
\resizebox{\columnwidth}{!}{
\begin{tabular}{| l| l| l| l| c |}
  \hline
  Dataset & Type & Features & Samples & Model\\
  \hline			
  Boer et al. 2010 & Metabolites & 101 & 136 & $X \sim Condition + Method + (1|Block) + \epsilon$ \\
  Hackett & Peptides & 11168 & 75 & $X \sim Condition + (1|Block) + \epsilon$\\
  Fell et al. 2007 & Peptides & 5279 & 448 & $X \sim Segregant + (1|BioRep) + \epsilon$ \\
  \hline
\end{tabular}
}
\caption[Summary of benchmark datasets]{Summary of benchmark datasets}
\label{ch-quant_anal:mass_spec_design}
\end{center}
\end{table}


\begin{figure}[h!]
\begin{center}
\includegraphics[width = 1\textwidth]{ch-quant_anal/Figures/MSanalysis_part1.pdf}
\caption[Schematic representation of peptide-level and metabolite-level analyses]{Schematic representation of peptide- and metabolite-level analyses}
\label{ch-quant_anal:MSworkflow_p1}
\end{center}
\end{figure}

The goal of this comparison is to determine whether mass spectrometry data can be assumed to follow a log-normal distribution; and if so, which variance components are relevant. To the extent that log-normality is appropriate when analyzing peptides and metabolites, powerful tools such as regression can be used, greatly simplifying downstream analysis. To the extent that log-normality may be a poor approximation of observed variability in mass spectrometry data, data analysis should use non-parametric methods that are robust to deviations from log-normality.

\subsection{Results $\&$ discussion}

Despite previous analysis, it is not immediately apparent whether feature-specific and signal-dependent uncertainty will grossly impact data analysis. Feature-specific variation is directly accounted for by estimating metabolite or peptide abundance \cite{Costenoble:2011hia, Boer:2010fb}, but this variation is regularly neglected when protein abundance is estimated through simultaneous regression of matching peptides \cite{Oberg:2012bm}. Regardless of whether feature-specific variation is assumed, its role has not been clearly discussed. Similarly, while low-signal measurements are consistently noisier than high-signal measurements (\hyperref[ch-quant_anal:replicate_corr]{Figure \ref{ch-quant_anal:replicate_corr}}) \cite{Zhu:2011jr,Oberg:2012bm,Navarro:2014ke}, it is unclear whether this effect is important when analyzing a single feature with a small range of measured ion counts. To assess the extent to which feature-specific and signal-dependent noise impact uncertainty in mass spectrometry data, three alternative variance models were fitted to each of the benchmark datasets (\hyperref[ch-quant_anal:modelComparison]{Figure \ref{ch-quant_anal:modelComparison}}). 

\begin{figure}[h!]
\begin{center}
\includegraphics[width = 1\textwidth]{ch-quant_anal/Figures/model_comparison.pdf}
\caption[Sources of variability in mass spectrometry]{Sources of variability in mass spectrometry. The included examples of peptide variance are simulations that represent the expected sources of uncertainty under each of the proposed variance models. Plots of signal-dependent variance are empirically derived by applying each variance model to the Fell et al. 2007 dataset.}
\label{ch-quant_anal:modelComparison}
\end{center}
\end{figure}

The feature-specific model assumes feature-wise homoschedasticity, and thus, can be evaluated through standard feature-wise regression. The ion count-dependent model assumes that feature-level variation reflects differences in signal intensity between features. To test this model, residuals were binned according to ion count, and the bin variance was determined as the mean of squared residuals, adjusted for fitted degrees of freedom (\hyperref[eq:spline_totalvar]{Equation \ref{eq:spline_totalvar}}). To determine the expected variance of a single observation based on its ion count, a spline function was estimated based on the relationship between average variance and intensity-binned ion count.

\begin{align}
\sigma^{2}_{ic,k} &= \frac{\sum_{k}^{K}\epsilon_{k}^{2}}{K}\label{eq:spline_totalvar}
\end{align}

The final evaluated model, the feature/ion count-dependent model, combines the two previous approaches; it assumes that features differ in their inherent noisiness and that on top of this variation, some observations have added noise due to low signal strength. This model was fitted similarly to the two simpler models. The excess variance of low ion count observations is found by subtracting the peptide-specific variance in a bin from the total variance and by fitting a spline (\hyperref[eq:spline_excessvar]{Equation \ref{eq:spline_excessvar}}).

\begin{align}
\sigma^{2}_{ic,k} &= \frac{\sum_{k}^{K}\epsilon_{k}^{2} - \sigma^{2}_{pep,k}}{K}\label{eq:spline_excessvar}
\end{align}

As a first test of model performance, we determined whether studentized residuals ($\epsilon_{ij} / \sigma_{ij}$) were approximately normally distributed (\hyperref[ch-quant_anal:normalityTests]{Figure \ref{ch-quant_anal:normalityTests}}). This analysis was done on a per-feature basis to allow for a summary of the fraction of features where a given variance model was appropriate. Each feature was summarized based on two tests of normality: the Shapiro-Wilk test \cite{Shapiro:1965gf} and the less powerful Kolmogorov-Smirnov test \cite{Lilliefors:1967bh}. In addition, kurtosis was calculated to determine the influence of a fat-tail/outliers. To summarize all features, kurtosis was averaged across all features. The two tests of normality were summarized based on the fraction of features that did not strongly deviate from normality ($\pi_{0}$) \cite{Storey:2003cj}. Additionally, to test the extent to which deviations from normality were driven by a minority of extreme observations, 0.1-10\% of the most extreme studentized residuals were removed and normality was reassessed.

The residual variation in the Boer metabolomic dataset is approximately normally distributed regardless of which variance model is adopted. It should be noted, however, that the removal of $\sim$1-2\% of the most extreme observations results in greater normality.  Removing more extreme observations than 1-2\% truncates the appropriate tails of the normal distribution, leading to lower kurtosis than would be expected from a univariate Gaussian ($\kappa = 3$). This results in extreme deviations from normality, as determined by the Shapiro-Wilk test. The residuals in the Fell proteomics dataset show pronounced deviations from normality based on both the Kolmogorov-Smirnov and Shapiro-Wilk tests and an excess of extreme observations, as determined by kurtosis. In this dataset, ion count-dependent variation substantially decreases the influence of outliers and improves the fraction of peptides that can be modeled as approximately normally-distributed. The residuals of the Hackett proteomics dataset are approximately normal regardless of the variance method, although removal of $\sim$3\% of extreme observations further improves normality.

\begin{figure}[h!]
\begin{center}
\includegraphics[width = 0.8\textwidth]{ch-quant_anal/Figures/normalityTests.pdf}
\caption[Evaluation of residual normality]{Evaluation of residual normality}
\label{ch-quant_anal:normalityTests}
\end{center}
\end{figure}

While residual normality is an indication that log-normality is generally an appropriate assumption, to fully make use of the determined variance, the residuals should follow a normal distribution, with variance specified by the variance model (\hyperref[eq:resid_normality]{Equation \ref{eq:resid_normality}}).

\begin{align}
\epsilon_{ij} &\sim \mathcal{N}(\mu = 0, \sigma = \sigma_{ij})\notag\\
\epsilon_{ij} / \sigma_{ij} &\sim \mathcal{N}(\mu = 0, \sigma = 1)\label{eq:resid_normality}
\end{align}

To determine which variance model is best supported on a feature-to-feature basis, the log-likelihood of residuals was calculated under each variance model (\hyperref[eq:resid_norm_likelihood]{Equation \ref{eq:resid_norm_likelihood}}), and the variance model that best fit each feature was revealed (\hyperref[ch-quant_anal:modelFits]{Figure \ref{ch-quant_anal:modelFits}}). For nearly all features, feature-specific variation is best able to model variability in the Boer metabolomics dataset. This is largely because ion count-dependent variation in metabolite abundance is a relatively weak signal that is swamped by between-metabolite differences in uncertainty.  In both proteomics datasets, the use of peptide-specific variation better fits some peptides; for others, the inclusion of additional ion count-dependent variation improved the accuracy of variance estimation. While peptide variance is strongly impacted by signal strength (\hyperref[ch-quant_anal:modelComparison]{Figure \ref{ch-quant_anal:modelComparison}}), within the bins used to estimate this added effect, individual residuals are on average less variable than the bin average. Thus, while in ideal cases, the inclusion of ion count-dependent variability can naturally include ``outliers,'' in other cases, applying weights to individual residuals only adds additional noise. 

\begin{equation}
\ell_{i} = \sum_{j}^{J}\text{ln}\mathcal{N}(x = 0; \mu = 0, \sigma = \sigma_{ij})\label{eq:resid_norm_likelihood}
\end{equation}

\begin{figure}[h!]
\begin{center}
\includegraphics[width = 0.6\textwidth]{ch-quant_anal/Figures/varianceModelPerf.pdf}
\caption[Relative support for tested variance models]{Relative support for tested variance models}
\label{ch-quant_anal:modelFits}
\end{center}
\end{figure}

Mass spectrometry data has great biological utility; but, in some cases, appropriately modeling its variation may be necessary before log-normality can be assumed. Estimating the variance of individual observations facilitates appropriate weighing of observations during regression to determine the main effects of interest.  The variable uncertainty of estimates of individual peptides can, in turn, be used to improve protein-level inference; multiple peptides that correspond to the same protein can be measured, and these different measurements can be integrated proportionally to their precision ($\sfrac{1}{\sigma^{2}}$) \cite{Navarro:2014ke}. This approach unlocks a powerful, novel approach to protein quantification and variance estimation (\hyperref[ch-quant_anal:pepToProt]{Figure \ref{ch-quant_anal:pepToProt}}), which is reported in \hyperref[proteomicsEM]{Appendix \ref{proteomicsEM}}. Briefly, this approach allows meaningful biological deviations from the standard peptides-to-proteins inference to emerge from the data. This is accomplished by appropriately assessing peptides that could belong to multiple proteins and by identifying peptides that strongly deviate from the overall trend of the protein, suggesting possible covalent-modification.

\begin{figure}
\begin{center}
\includegraphics[width = 0.8\textwidth]{ch-quant_anal/Figures/MSanalysis_part2.pdf}
\caption[Estimating protein abundance from peptides]{Estimating protein abundance from peptides}
\label{ch-quant_anal:pepToProt}
\end{center}
\end{figure}

