I am very grateful to the members of the Rabinowitz, Storey and Botstein groups as well as other members of the Lewis-Sigler Institute for providing contrasting perspectives that naturally support interdisciplinary thinking. Working in this environment has refined my scientific interests and set me on a path to scientific questions that inspire me.

I would like to begin by thanking my advisor Josh Rabinowitz whose support and direction have shaped my project immensely. Josh paralleled my interest in metabolic systems biology, providing motivation and direction towards a meaningful, shared goal. Josh is an excellent communicator, and has refined my scientific communication whether verbal, written or through visualization.

I would like to thank all past and present members of the Rabinowitz research group for their advice and support over the past 5 years. I would like to further highlight the formative contributions of several individuals: Vito Zanotelli provided enormous support by establishing a formalism for analyzing reaction kinetics, helping to assemble summaries of literature regulation and by implementing approaches to deal with flux uncertainty; Wenxin Xu has helped with verifying regulatory predictions and testing the validity of flux estimation methods; Chel Nofal has been an invaluable sounding board for evaluating the merit and delivery of results; Ian Lewis helped with the experimental measurement of media samples using NMR. I would also like to thank Greg Ducker, Lukas Tanner, Xiaoyang Su, Yifan Xu, Chris Crutchfield and Meytal Higgens for help with experimental methods.

Other members of the Lewis-Sigler Institute have enormously helped my work and development at Princeton: Josh Storey greatly reshaped my approach towards data analysis, arming me with many statistical tools for analyzing diverse types of data; David Bostein set me on the path to thinking about biology in a quantitative manner and helped me to understand the awesome power of yeast genetics; Pat Gibney taught me how use yeast genetics, he made several of expression strains used to test regulation, and provided invaluable experimental advice on all things yeast; Dave Robinson provided lots of programming advice and was always available to think about statistical questions with me; Jonathan Goya helped run many of the chemostat experiments, was an integral part of the proteomics analysis and helped me think about the quantitative relationship between proteins and transcripts; David Perlman provided lots of advice on setting up the proteomics experiments and spent a huge amount of effort on ensuring that the dataset was top notch; Sandy Silverman taught me how to do numerous yeast experiments. I would also like to thank Keyur Desai, Tina Hu and Peter Andolfatto for their guidance.

I would like to thank the great administrators whose helped me easily navigate bureaucratic challenges including Marybeth Fidele, Jen Brick and Ping Ge. 

I would also like to my past scientific mentors who set me on the path to a PhD by making me see the intrigue of science: As an undergraduate, Teresa Gunn helped turn my interest in genetics into an interest in scientific discovery; Andy Clark taught me to think about grand questions and how to interrogate them using appropriate large datasets; Tony Greenberg fostered my interest in statistics, programming (and Brazillian Jiu-Jtsu).

Finally, I would like to thank my family. As the son of two veterinarians, the interest in biology that they instilled in me provided direction through college and beyond. My fiance Maya helped to give me perspective during grad school and helped with editing.