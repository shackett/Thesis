I am very grateful to the members of the Rabinowitz, Storey and Botstein groups as well as other members of the Lewis-Sigler Institute for providing contrasting perspectives that naturally support interdisciplinary thinking. Working in this environment has refined my scientific interests and set me on a lifelong path of probing inspiring scientific questions.

I would like to begin by thanking my advisor, Josh Rabinowitz, whose support and direction have shaped my project immensely. Josh paralleled my interest in metabolic systems biology, providing motivation and direction towards a meaningful, shared goal. A masterful communicator, Josh has refined the scope of my scientific expression: verbally, visually and linguistically.

John Storey can also not go without acklowledgement; John introduced me to statistical methods and approaches of visualization which revolutionized my approach to research. Being a fly on the wall at his group meetings helped me think about questions in a statistical manner. 

I would like to thank all past and present members of the Rabinowitz research group for their advice and support over the past five years. I would like to further highlight the formative contributions of several individuals. Vito Zanotelli provided enormous support by establishing a formalism for analyzing reaction kinetics. He helped assemble summaries of literature regulation and implemented approaches to deal with flux uncertainty. Wenxin Xu has helped with verifying regulatory predictions and testing the validity of flux estimation methods. Chel Nofal has been an invaluable sounding board for evaluating the merit and delivery of results. Ian Lewis assisted with the experimental measurement of media samples using NMR. I am also extremely grateful to Greg Ducker, Lukas Tanner, Xiaoyang Su, Yifan Xu, Chris Crutchfield and Meytal Higgens for helping with experimental methods.

Other members of the Lewis-Sigler Institute have enormously helped my work and development at Princeton: David Bostein set me on the path to thinking about biology in a quantitative manner and helped me to understand the awesome power of yeast genetics. Pat Gibney taught me how use yeast genetics; he made several of the expression strains used to test regulation and provided invaluable experimental advice on all things yeast. Dave Robinson provided programming advice and was always available and ready to think about statistical questions with me. Jonathan Goya helped run many of the chemostat experiments, was an integral part of the proteomics analysis and helped me think about the quantitative relationship between proteins and transcripts. David Perlman provided a plethora of advice on setting up proteomics experiments and spent a huge amount of effort to ensure that the dataset was top notch. Sandy Silverman taught me how to do numerous yeast experiments. I would also like to thank Keyur Desai and Peter Andolfatto for their guidance.

I would like to thank the great administrators whose helped me easily navigate bureaucratic challenges, including Marybeth Fidele, Jen Brick and Ping Ge.

I would also like to sincerely thank my past scientific mentors, who set me on the path to a PhD by making me see the intrigue of science. As an undergraduate, Teresa Gunn helped turn my interest in genetics into an interest in scientific discovery. Andy Clark taught me to think about grand questions and how to interrogate them using appropriate large datasets. Tony Greenberg fostered my interest in statistics, programming (and Brazillian Jiu-Jitsu).

I would also like to thank my family. As the son of two veterinarians,  I was instilled with a keen interest in biology that provided me direction through college and beyond. Finally, I want thank my fianc\'{e}, Maya, whose peace of mind/presence gave me perspective throughout my graduate education and whose proofreading efforts will hopefully make the reading of this dissertation more enjoyable.

This research is based upon work supported by the U.S. Department of Energy, Office of Science, Office of Biological and Environmental Research (BER), under Award Number DE-SC0012461. This research was supported in part by an award from the Department of Energy (DOE) Office of Science Graduate Fellowship Program (DOE SCGF). The DOE SCGF Program was made possible in part by the American Recovery and Reinvestment Act of 2009.  The DOE SCGF program is administered by the Oak Ridge Institute for Science and Education for the DOE. ORISE is managed by Oak Ridge Associated Universities (ORAU) under DOE contract number DE-AC05-06OR23100.  All opinions expressed in this work are the author's and do not necessarily reflect the policies and views of DOE, ORAU, or ORISE.