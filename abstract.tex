Advances in genomics continue to make detection of large numbers of related biomolecules more routine. For example, simultaneous measurement of many mRNAs, proteins or metabolites xxx not a complete sentence xxx. While these �omic techniques summarize how the state of one type of biomolecule changes across conditions, they provide a one-dimensional view of cellular processes that emerge at the interface between classes of species. To bridge multiple �omic dimensions, approaches are needed that rationally integrate diverse types of data. In assessing this question, we are interested in testing the validity of models that posit some interaction between diverse chemical species. This process entails not only a rigorous analysis of individual �omic datasets, but also a procedure for posing alternative models and statistically evaluating the support for each model. Here, we present a quantitative, scalable strategy for revealing steady-state metabolic regulation by integrating metabolomic, proteomic, and flux measurements. This involves analyzing, on a reaction-by-reaction basis, whether fluxes across conditions are accounted for by a Michaelis-Menten relationship between enzymes, substrates, and potential regulator concentrations. We collected the required data for yeast growing at 25 different nutrient-limited steady states and applied this strategy to reveal the primary physiological flux control mechanisms for over 40 metabolic reactions, encompassing 34 instances of physiologically-relevant allosteric regulation. The identified regulation included classical feedbacks and unexpected cross-pathway connections. Quantitatively, half of flux control resided in substrates and one quarter in enzyme concentrations. For reversible reactions, the remainder resided largely in product levels, and for irreversible reactions, in allosteric effectors. Thus, metabolic activity is substantially self-regulated by metabolites themselves. Across the diverse growth conditions studied, strong changes in mRNA expression generally resulted in corresponding changes in protein abundance. This association between mRNA and protein expression changes, however, was far weaker than expected; the levels of many proteins departed markedly from their cognate transcript. These deviations were highly non-random, suggesting that post-transcriptional regulation has an important role in modulating the cellular response to nutrient availability. This work collectively provides focused examples of how the structures of complex systems can be interrogated to identify meaningful regulatory relationships.  And, it is at the interface between �omic datasets that these valuable relations emerge.