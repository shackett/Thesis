\section{Conclusion}

It seems inconceivable that human complexity could be reduced to a physiological model in which a multitude of genetic variants and environmental exposures could jointly inform numerous complex phenotypes. However, I firmly believe that the field is poised to create just such a model for a suitable microbe. This endeavor will require detailed experimental measurements of nearly all cellular species, a technical feat which is becoming increasingly feasible through advancements in sequencing, mass spectrometry and other high-throughput methods. These measurements alone can characterize the static state of an organism, but they cannot fully inform the relationships between chemical species that lead to the emergent property of physiology. 

To learn how species interact, one important strategy in genomics has been detecting species' physical interaction using library-based methods \cite{LiebermanAiden:2009jz, Fields:1989dm, Orsak:2012ci, Johnson:2007fh}.  While this approach suggests physical associations, with few exceptions \cite{Reynolds:2011gs}, it is unclear whether such associations are functionally significant \cite{Nandy:2010ej, Scheer:2011df}. 

To establish the relevance of interactions, physiological covariation can be used to suggest dependence, but this is a challenging model inference problem due to model degeneracy. To associate variables, extensive orthogonal variation must be generated through a combination of genetic and/or environmental perturbations \cite{Greenberg:2011jf}. Regardless of how many conditions can be practically assessed, however, the large space of possible interactions between variables will challenge analysis. To reduce complexity, core principles of molecular biology and biochemistry can serve as a scaffold to lay biological models. My thesis research espouses this philosophy, providing clear, practical examples of how experimental data can guide the bottom-up reconstruction of complex systems.

When integrating multiple types of data, it is all the more important both to generate each dataset in a way that maximizes signal and to analyze the resulting data using appropriate statistical methods. Without such consideration, the added noise from combining multiple datasets will mask meaningful biological signal. Alternatively, inappropriate methods could make it difficult to determine whether a model is consistent with the input data. To elaborate upon such practical considerations, during \hyperref[ch:quant_anal]{Chapter \ref{ch:quant_anal}} I discussed the analysis of mass spectrometry data.  Mass spectrometry is an ideal platform for quantifying important physiological variables, such as metabolites, peptides/proteins and lipids. In some cases, however, experimental data does not conform to the assumption of log-normality. In situations where a standard log-normal model is not valid, I discussed how heteroscedastic models of parametric variation may allow for broader application of log-normal models to mass spectrometry data. When analyzing variation in a single class of species, such as metabolites, deviations can be dealt with through non-parametric approaches. By contrast, when a dataset is meant to be integrated into systems-level models, summarizing data based on parametric assumptions is of great utility.  This is because when creating models it is important that uncertainty in model outputs correspond to uncertainty in model inputs. The multivariate delta method used in \hyperref[simmer_delta]{Section \ref{simmer_delta}} is particularly useful in this regard \cite{Lynch:1998vx}.

By generating a multi-omic dataset where four major classes of biomolecules and processes (transcripts, proteins, metabolites and fluxes) were quantified across 25 physiologically and metabolically heterogeneous states, I was able to investigate two classes of multi-omic interactions. Enzymes, metabolites and fluxes collide at the level of metabolic reactions, while the process of translation fundamentally links transcripts and proteins. In each case, appropriate experiments allowed me to reduce a genome-scale question to a set of subproblems, analyzed at a more focused and tractable scale (i.e. either a single metabolic reaction or a single gene).

\subsection{Understanding the structure and regulation of metabolism}

During \hyperref[ch:simmer]{Chapter \ref{ch:simmer}}, I used an integrative `omics approach to model metabolism by virtue of reaction-level interactions between metabolites, enzymes and consequent flux. For each of 44 reactions, an approximate model of reaction kinetics was inferred based on balancing the plausibility of each model, $Pr(Model)$, with each kinetic model's quantitative support, $Pr(Data | Model)$. This approach, termed SIMMER (\underline{S}ystematic \underline{I}nference of \underline{M}eaningful \underline{M}etabolic \underline{E}nzyme \underline{R}egulation), identified 34 instances of quantitatively supported yeast allostery.  Many of these regulatory mechanisms were strongly supported based on past literature; others could be experimentally validated through \textit{in vitro} biochemistry, and still other were likely false positives due to experimental insufficiencies.

\subsubsection{Characterizing physiological regulation in yeast metabolism}

We determined that for 44 of 56 reactions, simple Michaelis-Menten kinetics (including at most two regulators) could accurately relate metabolite and enzyme concentrations to reaction flux. This is impressive because mechanistic modeling of the kinetics of well-studied reactions rarely reduces to Michaelis-Menten kinetics; and often many regulators are thought to collectively regulate activity \cite{Hill:1977vm}. For the purposes of flux control; reaction forms do not need to be mechanistically principled, rather a reaction form must only account for an approximately correct relationship between species and resulting flux across the physiological conditions investigated \cite{Fell:1997wg}. Spanning the diverse conditions that we studied, Michaelis-Menten kinetics was generally appropriate. Additionally, even when multiple regulators were predicted to function, a single regulator usually dominates flux control even for regulatory hubs such as phosphofructokinase, pyurvate kinase and DAHP synthase.  

The possible regulators that we tested for each reaction were aggregated from reports across all domains of life, yet the most likely regulator for well-studied reactions generally agreed with what we already know about yeast.  Negative feedbacks of anabolic pathways are the best studied cases of metabolic regulation because the logic of these pathways is clear; an end-product inhibits the first committed step for its synthesis.  By allowing the supply of an end-product to sense how quickly it is utilized, metabolic control of pathway flux can be transferred from synthesis to utilization \cite{CornishBowden:1995fy}. Because these regulatory events are so predictable and the end-product inhibitor can generally be easily isolated, most of these regulatory interactions were demonstrated \textit{in vitro} over fifty years ago. It is, moreover, encouraging that instances of canonical end-product feedback are kinetically important across normal growth conditions.

Glycolysis has also been extremely well studied in yeast both because it is the central path by which nearly all energy and biomass is made and because researchers have sought to understand how metabolism is rewired to accomplish the diauxic shift \cite{Zampar:2013fr}. As such, glycolysis is one of the few pathways for which we have any knowledge of flux control.  In yeast, the rate of fermentation is controlled by transcriptional alteration of glucose uptake as well as by phosphofructokinase activity \cite{Cortassa:1994is, Pritchard:2002ft}.  The latter is thought to be impacted by several regulators, particularly by activation by fructose 2,6-bisphosphate \cite{Cortassa:1994is, vanEunen:2012cr}. While we did not measure fructose 2,6-bisphosphate in this study, activation by fructose 1,6-bisphosphate or inhibition by citrate or AMP was necessary to predict flux through phosphofructokinase. Because we could not test the role of fructose 2,6-bisphosphate, we cannot conclude whether any predicted regualtion is meaningful or merely reflects (anti)correlation with the unmeasured fructose 2,6-bisphosphate. Pyruvate kinase is also thought to regulate glycolysis; allosteric activation of pyruvate kinase by fructose 1,6-bisphosphate controls the concentrations of metabolites in lower glycolysis, allowing lower glycolysis to keep pace with upper glycolysis when cells are faced with fluctuating glucose concentrations \cite{Xu:2012gg}. We predict that fructose 1,6-bisphosphate could indeed activate pyruvate kinase, albeit a large class of correlated metabolites are also quantitatively supported.

Outside of canonical regulation of glycolysis and amino acid and nucleotide synthesis, less is known about metabolic regulation because the logic of how most reactions should be regulated is unclear. Consequently, hypothesis-driven research is less fruitful. Because we can computationally test many possible regulatory models and interpret novel predictions in the context of how they alter flux across conditions, SIMMER is a powerful way to identify novel regulation with little prior knowledge.  This approach was successfully used, for example, to discover that alanine is a physiological inhibitor of ornithine transcarbamylase (OTCase: Arg3).  Because OTCase and aspartate transcarbamylase (ATCase) compete for a common metabolite carbamoyl phosphate, inhibition of OTCase by alanine can favor production of pyrimidines when aliphatic amino acids are abundant.  This provides a logical route by which amino acid concentrations can favor rRNA synthesis and thereby use amino acid reserves for protein synthesis.  

\subsubsection{Coordination of metabolism and the proteome leads to nutrient-dependent growth control}

SIMMER indicates whether Michaelis-Menten kinetics is an adequate model of reaction kinetics and if not, which regulation (if any) is best supported. Model inference naturally entails assessing whether the physiological variability in all reaction species can collectively explain changes in flux. To partition the joint influence of all species into the marginal influence of each specie, I motivated the concept of \textit{metabolic leverage}, where the relative influence of each specie is governed by how much the specie varies and how this variation impacts flux (\hyperref[fig:metabolicLeverage]{Figure \ref{fig:metabolicLeverage}}). Summarizing reactions this way, two classes of reactions emerge. Variable flux through kinetically reversible reactions is primarily accomplished through changes in substrates and/or product concentrations, with only minor contributions due to changes in enzyme levels.  In contrast, variable flux through kinetically irreversible reactions is attributable to joint variation in substrates, enzymes and possibly regulators.

Common trends in metabolic leverage shed light on how yeast are able to control their metabolism to grow optimally in radically differing environments. From analysis of metabolic leverage, reversible and irreversible reactions serve distinct kinetic functions. Variable flux through reversible reactions is primarily driven by changes in substrate and product concentrations. Thus, these reactions serve as a bridge between irreversible reactions. Irreversible reactions are usually affected by one or more external control mechanisms: either changes in expression wired to the transcriptional/post-transcriptional regulation or metabolic regulation. This regulation helps to coordinate pathway flux and metabolic demand. From the analysis of metabolic leverage, it appears that the metabolome inherently yields an appropriate distribution of flux while the changes in protein abundance, while important, primarily serve to improve efficiency. This conclusion helps to explain the notorious difficulty of altering pathway fluxes by increasing the expression of a single or even multiple pathway enzymes \cite{CornishBowden:1995fy,Schaaff:1989dl,Hauf:2000vu,Fell:1997wg}. 

The importance of this metabolic control paradigm is elucidated when one considers how metabolic robustness is maintained in the face of genetic and or non-metabolic environmental perturbations \cite{Larhlimi:2011ds}. If enzymes that catalyze reversible reactions are perturbed, the balance of substrates and products shifts to appropriately counteract the induced effect \cite{Fendt:2010gr}. When the activity of irreversible reactions is altered, the metabolic consequences of this perturbation can be framed in light of how they are regulated. This regulation is summarized by the substrate/product, enzyme, regulator metabolic leverage ternary diagram (\hyperref[fig:metabolicLeverage]{Figure \ref{fig:metabolicLeverage}A}). For enzymes residing near the center of this diagram, altered enzyme activity would be partially absorbed by changes in substrate abundance \cite{Fendt:2010gr}. If this change is insufficient, then, in the case of feedback control, a mismatch between pathway production and pathway demands alters pathway metabolite concentrations until feedback control attenuates the changes in pathway production \cite{CornishBowden:1995fy}.

Although, the metabolic network generally appropriately tunes substrate levels to the pathway demands, there is a cost of such metabolic fine-tuning. Regulation of flux by metabolites either results from reaction reversibility or via the transfer of metabolic control \cite{CornishBowden:1995fy, Fell:1997wg}. Accordingly, metabolic regulation inherently requires the variable accumulation of metabolites. In microbes, the maintenance of metabolite pools that purely assist kinetics incurs a fitness cost because these pools could be incorporated into biomass to support growth. In some instances, this cost can be massive. For instances, if feedback-control on a first-committed step was ablated, accumulation of products would only attenuate excess pathway production as thermodynamic equilibrium was approached \cite{CornishBowden:1995fy}. The fitness cost of metabolomic fine-tuning helps to explain why many metabolic enzymes are under strong purifying selection \cite{Greenberg:2008uy}, despite the fact that only a minority of enzymes are strongly regulated through abundance or exert meaningful flux control. Due to this strong selection, most standing genetic variation only result in small changes in metabolite concentrations, while stronger variants which impact flux are probably maintained through balancing selection (\hyperref[ch:quant_analysis:mQTL]{Section \ref{ch:quant_analysis:mQTL}}).

In yeast, under natural conditions, metabolism appears to be well regulated. Cases of substantial dysregulation could be manifest as large changes in product concentrations (as is seen in the uracil auxotrophy), or if allostery was the primary determinant of reaction flux. Because most reactions are primarily regulated through substrate(s) and enzyme concentrations, allostery primarily fine-tunes approximately correct inputs, resulting in only minimal excess accumulation of pathway products

\subsection{Understanding how enzyme levels are regulated}

It is routine to treat transcript abundance as a surrogate for protein abundance and gene function. As illustrated in \hyperref[ch:pt_compare]{Chapter \ref{ch:pt_compare}}, systematic differences between transcripts and proteins, however suggest that it is unwise to assume such a false equivalence. Inspecting variation in proteins-per-transcripts reveals that post-transcriptional regulation through control of translational efficiency and/or protein degradation supplements the primary transcriptional response to nutrient availability. 

The basal level of protein expression control is through control of transcript abundance. The widespread adoption of microarrays and subsequently RNAseq determined under which conditions individual transcripts are elevated or depleted in many organisms \cite{Edgar:2002tt}. Nonetheless, we still have an incomplete knowledge of how transcript levels are determined. But, transcriptional control is increasingly understood based on the activity of transcription factors whose regulatory specificity is governed by cis-regulatory sequences \cite{Boorsma:2008cv, McIsaac:2012da}. Small RNAs further modulate expression by altering transcription and stability across conditions \cite{ValenciaSanchez:2006eb, Guo:2010tv}.

Post-transcriptional regulation is increasingly recognized as an important layer of control on top of transcript abundance. Still, most studies have only focused on between-gene regulatory variation rather than assessing how individual genes are regulated across conditions. Both average translational efficiency and protein stability vary greatly across genes \cite{Belle:2006hv, Ingolia:2011hu}. These processes can be altered across conditions;  translation rate can be impacted by both amino acid levels \cite{Klumpp:2009ic} and by modulation through the use of internal ribosome entry sites (IRESs) \cite{Hellen:2001cw}. Degradation through specialized pathways also depends on the environment \cite{Callis:1995ki}. Less work exists to bridges between-gene and between-condition variation. Doing so would entail assessing how different genes are regulated across conditions \cite{Brar:2012ig, Jovanovic:2015hp}. 

By comparing transcriptional changes to changes in protein expression across 25 diverse physiological conditions, transcriptional and post-transcriptional regulation can be studied under conditions that parallel natural environmental variation.  Studying each gene independently, I determined when nutrient-induced changes in protein abundance arose from changes in transcript abundance, post-transcriptional regulation, or merely measurement noise. This analysis showed that while strong transcriptional changes propagate into corresponding changes in protein abundance, additional layers of post-transcriptional control are relevant.

In modulating the primary transcriptional response, post-transcriptional regulation tailors the proteome to nutrient availability. While classical examples of transcription regulation indicate how metabolites can control transcription \cite{Jacob:1961du, Jones:1982dn}, the expression of most transcripts is more strongly affected by growth rate than by the availability of specific nutrients.  In comparison, the proteome appears more similar to the metabolome; protein levels are altered in response to the specific nutritional needs of the cell (\hyperref[fig:dataSummary]{Figure \ref{fig:dataSummary}A}). Metabolic effects on post-transcriptional regulation could account for such a divide; variable amino acid levels impact translation rate \cite{Klumpp:2009ic} and protein degradation responds to nitrogen starvation \cite{Zundel:2009dy, Xu:2013do}. By further linking metabolism to protein expression, post-transcriptional regulation serves an important physiological role in growth control.


