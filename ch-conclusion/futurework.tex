\section{Future Work}

The work that I have presented here can serve as a template for how metabolism and protein expression can be efficiently interrogated. My analysis of these questions, however, is limited by the absence of crucial information and an incomplete survey of physio-genetic variation. As such, when discussing future directions, I will primarily focus on how the caveats limiting my current research could be best addressed. From there, I will touch upon the real world utility of metabolic and expression models, specifically focusing on how such models could be applied to metabolic engineering to improve the efficiency of microbial compound production.

\subsection{Achieving a quantitative understanding of metabolism}

As discussed in \hyperref[ch:simmer]{Chapter \ref{ch:simmer}}, the kinetics of metabolic reactions can be analyzed \textit{en masse}. This is accomplished by measuring reaction flux and all reaction species across a sufficient array of conditions where alternative models can be statistically discriminated. The success of such an approach requires that one of the tested models approximates true \textit{in vivo} kinetics and that no measurements are systematically biased. In practice, no such guarantees are afforded at this juncture because a substantial list of caveats currently hinder SIMMER. These limitations largely arise from phenomenon with little demonstrated metabolic significance and unknown impacts. By experimentally addressing each question within the SIMMER framework, however, a more complete understanding of how metabolism is regulated can be found.

\subsubsection{Metabolomic incompleteness}

One current limitation of the SIMMER methodology is that while metabolomics is able to measure a meaningful fraction of important small molecules, this number is minute when compared to the total number of species in most metabolic reconstructions (e.g. 1,967 species reported in the yeast consensus reconstruction v7 \cite{BenjaminDHeavner:2013bp}). Mass spectrometry-based metabolomics is best at measuring abundant intracellular metabolites that readily ionize. This results in the preferential ascertainment of metabolites in central carbon metabolism, where, SIMMER can consequently be best applied. Because of the incompleteness of the ascertained metabolome, for most reactions, the primary substrate(s) have not been measured; and thus, SIMMER cannot be intelligently applied. In other cases, unmeasured products or minor substrates (such as protons or water) may prevent the identification of an appropriate kinetic model or lead to false positives. For example, a false positive may arise from the correlation between a missing product and a measured putative inhibitor. When such species are missing it is assumed that their concentration is invariant; this assumption, thus, will only impact the application of SIMMER to the extent that physiological variation in missing species impacts flux.

Measuring a wider array of metabolites would enable the broader application of SIMMER. A partial solution to this limitation will be naturally afforded through the passage of time; as mass spectrometer sensitivity continues to advance at a rapid pace \cite{Radionova:2015gs}, measurement of increasingly scarce species becomes more routine. Improvements in sensitivity are only a partial solution. Alternative protocols may be necessary to discriminate chemically similar species or to measure species that are not reliably detected using standard methods \cite{Lammerhofer:2013tz}.

Direct measurement of small molecules is the best solution to account for each small molecule's kinetic impact. Nevertheless, for some species measurement will be challenging due to their low concentration (e.g non-storage metabolites, outside of central carbon metabolism); lability (e.g. oxaloacetate); or measurement difficulty (e.g. protons and water). In some cases, such as for water, it might be reasonable to treat metabolites as invariant, but an interesting alternative solution to this limitation is the use of \textit{optimal regulators} (\hyperref[simmer_hyporeg]{Section \ref{simmer_hyporeg}}). Using this approach, both reaction kinetics and an optimal pattern of regulator abundance (informed by the metabolomics principle components) are simultaneously inferred. An analogous approach could be used to impute the abundance of an unmeasured metabolite, either for a single isolated reaction or for multiple coupled reactions.  For instance, the relative abundance of oxaloacetate could be inferred based on its shared involvement in mitochondrial malate dehydrogenase, citrate synthase and pyruvate carboxylase. Similarly, unknown proton concentrations could be assessed by aggregating numerous protonation reactions. 

While intriguing, jointly inferring the kinetics of coupled reactions detracts from the scalability of the SIMMER. A possible middle ground may be found by using an iterative approach. Such an approach would iterate between solving metabolite abundances using coupled reactions and then optimizing the kinetics of individual reactions conditioned on imputed metabolite abundance. On the more exploratory side, this PC-based imputation approach could be used to explore the kinetics of additional reactions. Multiple under-characterized reactions with unmeasured metabolites could be combined into a coupled inference problem. This could be a powerful approach for metabolic exploration; inference would still be underdetermined and the use of a bayesian approach would appropriately convey uncertainty in individual kinetic parameters and metabolite loadings. PC-based imputation may serve as a powerful gap-filling approach, not only in metabolism, but for elsewhere in genomics and beyond.

\subsubsection{Improving the precision of flux estimation}

In order to break metabolic pathways into reaction-level inference problems, reaction flux must be treated as a fixed quantity. Determining flux is challenging; fluxes are rates rather than physical species that can be directly measured. Techniques to determine these rates have either largely estimated fluxes that maximize growth using Flux Balance Analysis (FBA) \cite{Orth:2010hb} or introduced isotopically-labelled metabolites that propagate through metabolism using Metabolic Flux Analysis (MFA) \cite{Zamboni:2009bp}. The needs of inferring fluxes in chemostat culture are somewhat different from standard problems: cells grow at known rates and metabolic sources and sinks can be readily estimated. Using this information, the variant of Flux Balance Analysis (FBA) that I developed is well-suited for estimating fluxes that are strongly determined by these boundary fluxes. This method focuses on finding a least-squares solution that optimally conforms to experimental data by using quadratic programming, 

It is unlikely that all fluxes are well-estimated by virtue of metabolite uptake, excretion and incorporation into biomass components. Based on boundary fluxes alone, it is difficult to estimate flux through pathways that have stoichiometrically equivalent routes for flux (such as in 1C metabolism) or through pathways that primarily affect energy or reducing equivalents rather than uptake, excretion or biomass component synthesis. To better quantify such pathways, future work should supplement experimental boundary fluxes. These can be determined through spectrophotometry and $^{1}$H-NMR with additional experimental measurements to further constrain possible flux.

Respirometry could allow for better estimation of glycolytic and TCA flux is respirometry.  In microbes, oxygen consumption is a surrogate for electron transport chain flux. The respiratory quotient $\left(\sfrac{\nu_{CO_{2}}}{\nu_{O_{2}}}\right)$ is a strong indication of the balance of fermentation versus aerobic respiration \cite{Boer:2003fi,BARFORD:1979ei}. To further refine estimates of intracellular flux, techniques from MFA could be applied to chemostat culture. This could be particularly useful for determining the ratio of fluxes between pathways, such as between glycolysis and the oxidative pentose phosphate pathway, or between lower-glycolysis and the TCA cycle \cite{Sauer:2006ii, Jazmin:2013fg, Kromer:2014wr}. MFA is generally considered a comprehensive framework for the estimation of metabolism-wide fluxes that can incorporate experimental boundary fluxes. But, estimates of reaction flux could also be incorporated into a quadratic programming based method.

Estimating additional reaction fluxes using respirometry \&/or MFA would allow for better estimation of the expected values of pathway fluxes; and, the uncertainty in these measurements could also be used to better characterize the uncertainty in metabolism-wide flux. The quadratic penalty is constructed by weighting least-squared deviations by experimental precision analogously to Gaussian log-likelihood $\left(\nu^{T}\Sigma^{-1}\nu = \sum_{j = 1}^{J}\frac{\left(x_{j} - \mu_{j}\right)^2}{\sigma^{2}_j}\right)$. Accordingly, uncertainty in flux can be assessed by the maximum and minimum flux that could go through each reaction at the solution minimum as well as at sub-optimal solutions that approximate cuts through the likelihood distribution (\hyperref[conclusion-fluxUncertainty]{Figure \ref{conclusion-fluxUncertainty}}). This is conceptually similar to estimating confidence intervals based on the likelihood ratio. SIMMER currently deals with uncertainty in flux using flux variability analysis \cite{Mahadevan:2003wq}, based on how constrained individual fluxes are at the solution optimum. A solution that more fully addresses uncertainty could be readily incorporated into SIMMER by integrating over the full flux probability density function.  This solution was not used in \hyperref[ch:simmer]{Chapter \ref{ch:simmer}} because the absence of internal fluxes prevented strong constraints on most fluxes at sub-optimal values. This resulted in pathological flux distributions. 

\begin{figure}[h!]
\begin{center}
\includegraphics[width=0.8\textwidth]{ch-conclusion/fluxVar.pdf}
\caption[Procedure for accounting for flux uncertainty]{Procedure for accounting for flux uncertainty. \textbf{A)} The relationship between the maximum and minimum flux that are possible at a given quadratic penalty / log-likelihood can be used to construct a \textit{de facto} log-likelihood distribution for each reaction's flux. \textbf{B)} The log-likelihood distribution corresponds to a probability distribution, p(v).}
\label{conclusion-fluxUncertainty}
\end{center}
\end{figure}



\subsubsection{Mischaracterization of enzyme activity and kinetic differences between isoenzymes}

When incorporating enzyme-level information into SIMMER, measurement of protein relative abundance can be accurately applied in a simple case: (1) a single gene produces an enzyme, (2) each copy of this enzyme is kinetically indistinguishable. Outside of this simple scenario, kinetic inference is complicated by multiple distinct pools of enzymes that catalyze a single reaction. Relevant enzymatic pools with greatly varying kinetics include: isoenzymes that catalyze the same reaction \cite{Wilson:2003cb}, enzyme complexes where pools of active enzyme may not be reflected by the abundances of monomers \cite{Cohen:2000bj}, and post-translationally modified enzymes with altered metabolite affinity or activity \cite{Cohen:2000bj, Schulz:2014eo}. 

Because many enzymes do not operate as monomers, the abundance of each catalytically-active complex that catalyzes a given reaction should be included as a separate distinct ``enzyme'' in SIMMER. We have an incomplete understanding of how complex abundance and activity relate to the abundance of individual components. In general, multimers are stable, while a stoichiometric misbalance between components results in degradation of the excess \cite{Marianayagam:2004ie}. In such simple examples, the abundance of a complex would reflect the abundance of each component monomer.  The genome scale model used contains annotated complex membership \cite{BenjaminDHeavner:2013bp}, so to date, complex abundance has been estimated using the precision-weighted mean over all complex components. In some cases such a simplified view of complex formation may be inaccurate.  More sophisticated models of complex formation that estimate complex abundance from proteomics data could easily be incorporated into SIMMER, however.
 
When multiple complexes or isoenzymes catalyze a single reaction, we are left with two options; we can either assume that each set of enzymes is kinetically similar or different. If we assume that all pools of enzymes interact with metabolites in a similar manner (shared $k_{d}$ and possible $k_{i}$/$k_{a}$), the occupancy of each enzyme will be the same ($\sfrac{\nu}{\nu_{max}}$). $k_{cat}$ will vary to reflect that measured enzyme concentrations are relative and sizes of enzyme pools may differ. This is the standard approach currently used in SIMMER. This minimizes the total number of fitted kinetic parameters; and in most cases, a single complex likely carries the bulk of flux. If we chose to relax this assumption, SIMMER is already  compatible with reaction forms in which one or more kinetic parameters (besides $k_{eq}$) differ between isoenzymes or complexes. Treating simple Michaelis-Menten kinetics catalyzed by two isoenzyme (E$_{a}$ and E$_{b}$) as an example, reaction flux would follow \hyperref[Eq:isoenzymeMM]{Equation \ref{Eq:isoenzymeMM}}. DAHP synthase (Aro3/Aro4), the first step of aromatic amino synthesis, is one clear example in yeast where such isoenzyme-specific regulation is likely relevant; Aro3p is specifically inhibited by phenylalanine, while Aro4p is specifically inhibited by tyrosine \cite{Schnappauf:1998ec}. The kinetics of this reaction tested using isoenzyme-specific regulation, but the support for this regulation was quantitatively weaker than predicted regulation by phenylpyruvate (a phenylalanine precursor) (\hyperref[signif_regulators]{Table \ref{signif_regulators}}). 

\begin{equation}
\nu = k_{cat}^{E_{a}}\left[E_{a}\right]\frac{\left[S\right]}{\left[S\right] + k_{m}^{E_{a}}} + k_{cat}^{E_{b}}\left[E_{b}\right]\frac{\left[S\right]}{\left[S\right] + k_{m}^{E_{b}}}\label{Eq:isoenzymeMM}
\end{equation}

Pools of enzymes may differ kinetically because they are composed of different proteins, but the kinetics of otherwise identical enzymes or complexes may also differ if they possess different post-translational modifications.  Post-translational modifications (particularly phosphorylation) are rampant in yeast \cite{Fiedler:2009hx}; their effects on metabolism, however, are largely unknown aside from a few phosphorylation events with an implicated kinetic role, particularly in the TCA cycle \cite{Schulz:2014eo}. While post-translational modifications have not been characterized under the measured conditions, SIMMER could be easily adapted to suggest the physiological role of post-translational modifications in a similar manner to testing allostery. To illustrate how such inference could be conducted, we can consider whether assessing the effect of a single phosphorylation event with a putative effect on enzyme activity explains measured flux significantly better than a model without regulation. If we knew the relative abundance of both the modified protein, $\left[E_{p}\right]$, and unmodified protein $\left[E_{u}\right]$, reaction kinetics would be described by \hyperref[Eq:ptmMM1]{Equation \ref{Eq:ptmMM1}}. This equation fits one more parameter than the minimal model.

\begin{equation}
\nu = k_{cat}^{p}\left[E_{p}\right]\frac{\left[S\right]}{\left[S\right] + k_{m}} + k_{cat}^{u}\left[E_{u}\right]\frac{\left[S\right]}{\left[S\right] + k_{m}}\label{Eq:ptmMM1}
\end{equation}

In general, we will not have measured both the modified and unmodified protein. Rather, we will have measured the abundance of a covalently-modified peptide, $\left[E^{p}_{c}\right]$, and the relative abundance of the whole protein ($\left[E^{p}_{t}\right]$: unmodified plus modified). With this input data, we cannot distinguish scenarios where the maximum fraction of covalent modified peptides is 2\% of the total protein, from when 50\% of the protein is modified. To deal with this unknown, the maximum labelling fraction, $\phi_{max} \in [0, 1]$ must be estimated concurrently with kinetic parameters that post-translational modifications are posited to effect (\hyperref[Eq:ptmMM2]{Equation \ref{Eq:ptmMM2}}).

\begin{align}
c^{*} &= \underset{c\in C}{argmax} \sfrac{\left[E^{p}_{c}\right]}{\left[E^{t}_{c}\right]}\notag\\
\phi_{c} &= \phi_{max}\left(\frac{\sfrac{\left[E^{p}_{c}\right]}{\left[E^{t}_{c}\right]}}{\sfrac{\left[E^{p}_{c^*}\right]}{\left[E^{t}_{c^*}\right]}}\right)\notag\\
\nu_c &= k_{cat}^{p}\left[E_c^{p}\right]\frac{\left[S_c\right]}{\left[S_c\right] + k_{m}} + k_{cat}^{t}(1-\phi_c)E_{c}^{t}\frac{\left[S_c\right]}{\left[S_c\right] + k_{m}}\notag\\
 &= \left(k_{cat}^{p}\left[E_c^{p}\right] + k_{cat}^{t}(1-\phi_c)E_{c}^{t}\right) \frac{\left[S_c\right]}{\left[S_c\right] + k_{m}}\label{Eq:ptmMM2}
\end{align}

\subsubsection{Is the ``well-mixed'' assumption appropriate?}

Biochemistry is greatly simplified when we assume that processes are spatially uniform and thus ``well-mixed.'' Assuming that all cells within a culture are effectively equivalent is often an obligate requirement to collect enough physical material to quantify biological species. The use of many cells decreases the noise inherent in individual cells but removes information about how cells meaningfully differ. 

In extreme cases, culture-wide behavior may not reflect the behavior of individual cells. As an example, consider a metabolic reaction: A + B $\rightarrow$ C. If the metabolite A is present in only half of the cells, and B is present in the complementary half, C will not be produced in either cell population. However, applying the culture-wide well-mixed assumption interaction between A and B would be possible, allowing for the errant synthesis of C. An absence of \textit{in vivo} colocalization can lead to the same fallacy when either cellular compartmentalization is ignored or enzymes \&/or metabolites are erroneously assumed to be evenly distributed within a compartment.

In some cases, the assumption of cell-level similarity is likely appropriate, at least to a first approximation. The best-case scenario for such an assumption to hold is likely during chemostat growth, during which many generations of similar environmental conditions minimize the impact of environmental heterogeneity and historical contingency. The culture-wide reproducibility of chemostats suggests that the biological variance between individual cells is minimal. Although it is routine to verify the population is morphologically similar, at least in terms of cellular volume, during chemostat culture. While chemostat growth may be a best-case scenario for minimizing cell-to-cell variation, both random \cite{BarEven:2006dz, Kaern:2005gr} and non-random variation at the single cell level may be important at the population-level.

The clearest example of a process which varies between cells is the temporal evolution of the mitotic cell-cycle whose sinusoidal oscillations could not be summarized by a bulk-sum analysis \cite{Hartwell:1974uy, Spellman:1998wj}. Such temporally variable processes are not readily observed unless cellular population are coordinated \cite{Hartwell:1974uy, Tu:2006cl} or single cells are tracked through morphology \cite{Herskowitz:1988ut}, microscopy \cite{Venturelli:2015ec} or single-cell sequencing \cite{Patel:2014dt}. Each of these latter techniques has revealed that otherwise similar cellular populations are frequently composed of distinct clusters of cells. Nonetheless, the scope of such heterogeneity remains unclear. If single-cell concentrations of reaction species were available, SIMMER could be applied to model flux heterogeneity, but gaining the necessary input data is not currently possible. While advances in microscopy and automation enable the simultaneous tracking of larger numbers of proteins \cite{Ghaemmaghami:2003ds, Dubuis:2013cw}, understanding of single cell metabolic heterogeneity is currently limited by inadequate methods to detect metabolite levels of single cells \cite{Zenobi:2013il}.

As previously discussed, it is routine to measure the culture-wide abundance of biological species without accounting for intercellular variability. In general, such measurements are also carried out without regard for intracellular heterogeneity, either at the scale of compartments or even finer-scale localization. Eukaryotic cells are composed of distinct compartments with their own complement of metabolites and proteins, which jointly govern the compartment's metabolic activity. Within each compartment, metabolic activity may be further impacted by colocalization of enzymes, allowing for efficient metabolic channeling of metabolites between sequential enzymatic steps \cite{Ovadi:1995wy}. To appropriately account for intracellular heterogeneity, metabolic species should be measured at their \textit{in vivo} concentration. Perhaps intracompartmental variable concentrations could be systematically incorporated into genome-scale models through appropriate microscopy experiments. Such concerns are secondary, however, to accounting for compartment-wise metabolic variation.

The absence of compartment-scale resolution to metabolic data does not necessary greatly impact SIMMER. Let us consider the concentration of a metabolite that is only found in a single compartment. If the compartments' volume is proportional to intracellular volume, then the metabolite's intracellular concentration is proportional to an absolute abundance measurement. This results in an inferred kinetic constant off by a multiplicative constant, although the \textit{in vivo} reaction kinetics can still be determined. If, however, metabolites are differentially partitioned across conditions (between the cytosol, mitochondria and vacuole in particular), relative concentrations in compartments will not reflect that of whole-cells \cite{Kitamoto:1988wc}.

Differential partitioning could explain why phenylpyruvate was erroneously predicted as an inhibitor of DAHP synthase rather than the true regulators phenylalanine and tyrosine \cite{Schnappauf:1998ec}. Phenylalanine is partially stored in the vacuole, while phenylpyruvate is likely primarily cytosolic \cite{Kitamoto:1988wc}. Because phenylalanine is made from phenylpyruvate in the cytosol, phenylpyurvate may be a better proxy for cytosolic phenylalanine concentrations than whole-cell measurements. In addition to the vacuole, estimating the concentrations of mitochondrial metabolites is also challenging. This is because relative mitochondrial volume may vary across conditions, as suggested by the strong covariation of mitochondrial enzymes (Supplemental Figure 2) and by the fact that many mitochondrial metabolites are also present in the cytosol but do not readily mix.  

To better address the problems apparent when compartmentalization is ignored, in eukaryotes, each major metabolic compartments should be treated with the same care as with separate experimental conditions. Techniques are emerging that allow for analysis of both metabolomics and proteomics of individual sub-cellular compartments \cite{Klie:2011kq, Wuhr:2014fr}. Because these methods are relatively gentle, it is unclear whether such approaches could be used to quantify the mitochondrial metabolome due to its great level of activity. Such methods may be necessary to kinetically model the yeast TCA cycle. A clearer investigation of compartment-specific biomolecules should be complemented by microcopy-based approaches for measuring compartment volumes \cite{JENSEN:1993bz, Ghaemmaghami:2003ds}.

\subsubsection{Evaluating modeling assumptions}

I have previously discussed the numerous ways in which culture-wide measurements of reaction species and modeling assumptions may limit our ability to accurately determine reaction kinetics. These limitations are primarily experimental rather than conceptual. Measurements of post-translational modification and compartment-specific quantification of metabolite abundances could be readily incorporated into SIMMER. Insufficient data greatly limits the number of reactions whose kinetics can be interrogated. For other reactions, our inability to describe reaction kinetics despite measuring the primary reaction species  and possible regulators suggests that other factors are kinetically important. Missing data could limit our ability to identify appropriate reaction kinetics or could even lead to false-positive predictions of regulation. Nevertheless, our ability to accurately predict reaction flux through numerous reactions using only reaction species and regulators suggests that our simplified models are more often appropriate than not.

From my analysis, for 44 of 56 tested reactions, simple Michaelis-Menten kinetics including at most two regulators can accurately relate metabolite and enzyme concentrations to reaction flux. This is impressive firstly due to the large number of reasons why this could fail. Secondly, mechanistic modeling of the kinetics of well-studied reactions rarely reduces to Michaelis-Menten kinetics, and often many regulators are thought to collectively regulate activity \cite{Hill:1977vm}.  Under assay conditions, significant deviations from Michaelis-Menten kinetics may exist and the activity of regulators can be demonstrated. For the purpose of understanding flux control, a reaction form does not need to be principled mechanistically; rather, it must only account for an approximately correct relationship between species and resulting flux across the physiological conditions investigated \cite{Fell:1997wg}. 

When trying to estimate the kinetics of reactions that cannot be described using metabolites, enzymes and flux, additional experimental data will surely shed light on the role of unaccounted for sources of regulation. In other cases, however, inferring appropriate reaction kinetics may require alternative modeling assumptions rather than additional data. Such modeling assumptions could be easily relieved; in principle, using SIMMER any relationship between reaction species and resultant flux can be evaluated using flux. If an alternative reaction mechanism could be significantly supported over alternative contenders, this would suggest validity analogously to how reaction mechanisms are tested \textit{in vitro} \cite{CornishBowden:2012wb}. Other kinetic phenomenon such as cooperativity \cite{Hill:1910vo, Bush:2012jf} can also be tested, as was demonstrated when evaluating allosteric cooperativity.

\subsubsection{Application of SIMMER using more comprehensive data, broader conditions and new organisms}

Because we have characterized a reproducible set of conditions that contains much of meaningful metabolic variation in yeast, I hope this dataset will serve as an important scaffold for future work. In order to better understand yeast metabolism, it will be important to refine and expand our input data in the ways that I have discussed above. It will also be crucial to expand the number of experimental conditions used to study kinetics. Investigating additional conditions is important for two reasons. First, when too few conditions have been tested, alternative models of reaction kinetics may not be discriminated, for instance, due to correlation of putative regulators. Second, we can only identify important regulation that impacts the conditions that we have studied. By expanding the number of conditions that we investigate, we can unmask regulation that is important in these conditions. Furthermore, SIMMER can identify this regulation by virtue of the kinetic importance it gains in these new conditions.

To improve kinetic inference when choosing additional conditions to study, we would ideally maximize the number of distinct input-output pairs that can found (i.e. combinations of flux and all potential reaction species). We would also prioritize conditions spanning meaningful metabolic transitions. These two goals are broadly compatible; diverse conditions generate extensive multi-`omic variation and reveal metabolic regulation necessary to operate at these states.

For example, if we grew yeast using ethanol as a carbon source rather than glucose, this would result in a unique pattern of flux, enzymes and metabolites in addition to the emergence of conditionally-meaningful regulation. This would allow us to confirm the regulation that facilitates the glycolytic to gluconeogenic switch \cite{Zampar:2013fr}. It also would inform regulation more broadly through the addition of a new metabolic state.

Similar logic could be used to better understand the specific metabolic impact of regulators of protein abundance. These regulators could be important transcription factors such as Gcn4 or affect signaling pathways such as Ras.  For example, a hypoactive variant of the yeast Ras-regulator Ira2 greatly impacts transcript and protein abundance, but also alters the levels of glycolytic metabolites and flux (\hyperref[ch:quant_analysis:mQTL]{Equation \ref{ch:quant_analysis:mQTL}}) \cite{Breunig:2014bu}. Through SIMMER's connection of $\left[regulator \rightarrow enzymes \rightarrow metabolites \rightarrow flux\right]$, the specific manner in which hypoactive Ira2 impacts flux could be discerned from ancillary changes in enzyme and metabolite. 

While SIMMER is a powerful way to understand yeast metabolism, its application to a well-studied, relatively simple organism does not fully highlight SIMMER's utility. Yeast metabolic regulation has been studied through decades of \textit{in vitro} biochemistry and genetics; however, such time-intensive characterization is not feasible for the majority of organisms. While previously reported \textit{in vitro} regulators can be prioritized and physiologically validated using SIMMER, in general, such prior assumptions are unnecessary or can be supplanted by regulation favored through genetic conservation.

When applying SIMMER to new organisms, or reactions with little or no previous chemical interrogation, the use of optimal regulators is a powerful way to prioritize regulatory metabolites. The reductionist approach of SIMMER will, moreover, be greatly useful when investigating the metabolic regulation of more complicated organisms, including humans. In higher eukaryotes, flux inference based on steady-state assumptions and parsimony largely break down; these methods are primarily supplanted by pathway-specific methods based on the dynamic turnover of isotopically-labelled metabolites. The scalability of SIMMER could help to elucidate the regulation and kinetics of these pathways. Because measurement of flux are routinely complimented with metabolomics and the generation of mutants that perturb pathway activity and metabolites, SIMMER could be applied largely with existing data.

\subsubsection{Translating reaction-level kinetics into metabolism-level predictions}

Determining reaction kinetics is important because it tells us how each reaction in a complex dynamical system will behave as function of its inputs. This is of massive importance in metabolic engineering. To increase flux through a pathway of interest that produces a commodity chemical, ideally we would know how pathway flux responds to changes in pathway enzymes. Due to the difficulty of reaction form inference, current models of metabolic regulation are limited by an incomplete understanding of metabolic regulation and reaction kinetics. Such models may lead to predicted metabolite-enzyme-flux relationships which agree with the data on which they are fitted. They rarely, however, generalize when applied to new conditions. Because of these inaccuracies, it is difficult to predict how pathway flux will respond to genetic perturbations and even more challenging to understand how the metabolic network (and, indeed, broader regulation) will respond to such a perturbation. 

Due to these limitations, progress in metabolic engineering has generally been born out of brute force trial and error. Understanding how metabolism responds to changes in enzyme abundance is well-studied in Metabolic Control Analysis (MCA). Unfortunately, MCA generally requires a complete kinetic description of all relevant reactions; so, to date, its application has been largely impractical \cite{Kacser:1973fe, Fell:1997wg}. By facilitating the estimation of reaction forms, SIMMER is an ideal tool to inform MCA-based pathway models. To demonstrate the application of SIMMER to MCA, I generated an MCA-based model of glycolysis, which reproduces two primary aspects of yeast glycolytic control (\hyperref[fig:MCA]{Figure \ref{fig:MCA}}).

Similar methods can be used to predict changes in flux in response to genetic perturbations or the removal of allostery. This would entail introducing an instantaneous change in a parameter (enzyme abundance or removing allostery) and then following the time-evolution of the system as it reaches a new steady-state. Each reactions form is fitted across a wide range of metabolic conditions and contains relevant regulation. Accordingly, the behavior of individual steps should accurately respond to moderate changes, resulting in reasonably accurate predictions of pathway flux.

\subsection{Towards a comprehensive model of protein expression}

By analyzing systematic deviations between proteins and transcripts in \hyperref[ch:pt_compare]{Chapter \ref{ch:pt_compare}}, I suggested that post-transcriptional regulation provides a highly relevant layer of protein-expression control. This analysis was not ideal due to differences in experimental design between the two compared datasets. By measuring proteins and transcripts in the same culture, and ideally accounting for biological variance through replication, a more faithful comparison between these species could be conducted.  Such an analysis could also disambiguate the role of translational efficiency and protein degradation through experimental determination of one (or both) of these processes using existing methods \cite{Ingolia:2009dp, Belle:2006hv}.

The central dogma of molecular biology indicates that information in a cell largely flows from DNA to RNA to protein. The nearly universal relationship is limited because it is non-quantitative. Meanwhile, the composition of the proteome is governed by the variable rate of information transfer both between-genes and across-conditions. Towards this end, a quantitative central dogma of molecular biology is needed to account for the relationship between changes in protein expression and the intra- and extra-cellular environment.

Such a quantitative model of protein expression could be directly tied into models that depend on protein activity, such as metabolism \cite{OBrien:2013fl}.  One essential challenge when trying to predict the behavior of metabolism is that an induced change in protein expression will result in a change in flux and metabolite abundance. This will, consequently result in a secondary change in gene expression. This change could be due to the cost of over-expression \cite{Dykhuizen:1987uq}, impacts of nutrient sensing on transcriptional \cite{Brauer:2008jn} and post-transcriptional control (\hyperref[ch:pt_compare]{Chapter \ref{ch:pt_compare}}), or effects such as osmolarity and pH \cite{Csonka:1991wf}. When there are unknown responses to a genetic or environmental perturbation the accuracy of metabolic prediction declines as metabolism is pushed beyond measured conditions.

If an integrated expression-metabolism model were possible, metabolic behavior could be predicted as a function of the environment and expression. Expression could, in turn, be predicted based on environmental and metabolic state. The industrial impacts of this development would be massive. By reliably connecting genetic and environmental changes to their resulting metabolic impact, pathway fluxes could be efficiently altered; such fluxes could produce either native compounds or favor synthetic pathways that synthesize foreign compounds. In this case, pathway flux would operate as a tunable knob. Balancing flux into a desired pathway with fluxes that fulfill an organism's biosynthetic requirements would allow for predictable steady-state synthesis. 

Predicting metabolic behavior based on an organism's genetic and environmental state is the key to unlocking the industrial capabilities of microbes. Establishing this link between genotype and phenotype through context-dependent mechanistic models is the same essential connection that is sought when investigating the genetic basis of complex traits and disease. Expanding this connection to higher-level traits could be accomplished using the same principles as those used to dissect the complexity of yeast. When studying disease, models of cellular behavior must first be improved. Next, models relating organ-level behavior must be developed. Ultimately, these systems must be integrated under that auspices of medicine, itself an endeavor which has generated \textit{de facto} models of disease occurrence. 
