\section{Future Work}

While the work I have presented can serve as a template for how metabolism and protein expression can be efficiently interrogated, my analysis of these questions is limited by the absence of crucial information and an incomplete survey of physio-genetic variation. As such, when discussing future directions of fruitful research I will primarily focus on how the caveats limiting my current research could be best addressed. From there I will touch upon the real world utility of metabolic and expression models, specifically focusing on how such models could be applied to metabolic engineering to improve the efficiency of microbial compound production.

\subsection{Achieving a quantitative understanding of metabolism}

As discussed during \hyperref[ch:simmer]{Chapter \ref{ch:simmer}}, the kinetics of metabolic reactions can be analyzed \textit{en masse} if we have measured reaction flux and all reaction species across a sufficient array of conditions where alternative models can be statistically discriminated. The success of such an approach requires that one of the tested models is an adequate reflection of true \textit{in vivo} kinetics and that no measurement is systematically biased. In practice, no such guarantees be afforded at this juncture as a substantial list of caveats currently adulterate SIMMER. These limitations largely arise from phenomenon with little demonstrated metabolic significance and thus their specific impacts are largely unknown, however by experimentally addressing each question within the SIMMER framework, physiological significance could be demonstrated.

\subsubsection{Metabolomic incompleteness}

One current limitation of the SIMMER methodology is that while metabolomics is able to measure a meaningful fraction of important small molecules, this number is minute when compared to the total number of species in most metabolic reconstructions \textcolor{red}{look up}. Mass spectrometry based metabolomics is best at measuring abundant intracellular metabolites which readily ionize resulting in the preferential ascertainment of metabolites in central carbon metabolism, where consequently SIMMER can be best applied. Because of the incompleteness of the ascertained metabolome, for most reactions, the primary substrate(s) have not been measured and thus SIMMER cannot be intelligently applied. In other cases unmeasured products or minor substrates, such as protons or water, may prevent the identification of an appropriate kinetic model or lead to false positives, such as if a missing product is correlated with an inhibitor.  When such species are missing it is assumed that their concentration is invariant; thus this assumption will only impact the application of SIMMER to the extent that physiological variation in missing species impacts flux.

Measuring a wider array of metabolites would allow the broader application of SIMMER. A partial solution to this limitation will be naturally afforded through the passage of time; as mass spectrometer sensitivity continues to advance at a rapid pace, measurement of scarcer and scarcer species becomes more routine. Improvements in sensitivity are only a partial solution as alternative protocols may be necessary to discriminate chemically similar species, or to measure species that are not reliably detected using standard methods. 

While direct measurement of small molecules is the best solution to account for each small molecule's kinetic impact, for some species measurement will be challenging due to either their low concentration (e.g \textcolor{red}{fubar}), lability (e.g. oxaloacetate), or measurement difficulty (e.g. protons and water). In some cases, such as for water, it might be reasonable to treat such metabolites as invariant, but an interesting alternative solution to this limitation is motivated by use of \textit{hypothetical regulators} during \textcolor{red}{cite section}.  Using this approach, both reaction kinetics and an optimal pattern of regulator abundance (informed by the metabolomics principle components) are simultaneously inferred. An analogous approach could be used to impute the abundance of an unmeasured metabolite, either for a single isolated reaction, or for multiple coupled reaction.  For instance, the relative abundance of oxaloacetate could be inferred based on its shared involvement in mitochondrial malate dehydrogenase, citrate synthase and pyruvate carboxylase. Similarly the concentration of protons in the cell, whose variation across physiological conditions is largely unknown, could be assessed by coupling the numerous cellular protonation reactions. While intriguing, jointly inferring the kinetics of coupled reactions takes away from the scalability of the SIMMER, a problem that could likely be relieved using an Expectation-Maximization-like approach to iterate between solving metabolite abundances through reaction-coupling and optimizing the kinetics of individual reactions conditioning on imputed metabolite abundance. On the more exploratory side, this PC-based imputation approach could be used to explore the kinetics of additional reactions by combining multiple under-characterized reactions with unknown metabolites together into a coupled inference problem. This could be a powerful approach for metabolic exploration as the inference would still be underdetermined and the use of a bayesian approach appropriately conveys uncertainty in individual kinetic parameters and metabolite loadings. Beyond metabolism, elsewhere in genomics and beyond, PC-based imputation may serve as a powerful gap-filling approach.

\subsubsection{Mis-characterization of enzyme activity and kinetic differences between iso-enzymes}

When incorporating enzyme-level information into SIMMER, measurement of protein relative abundance can be accurately applied in the simple case where a single gene produces an enzyme, and each copy of this enzyme is kinetically indistinguishable. Outside of this simple scenario, kinetic inference is complicated by multiple distinct pools of enzymes that catalyze a single reaction. Relevant enzymatic pools whose kinetics could greatly vary include isoenzymes which catalyze the same reaction \textcolor{red}{cite bradley}, enzyme complexes where pools of active enzyme may not be reflected by the abundances of monomers \textcolor{red}{cite}, and cases where post-translational modification of enzymes alters metabolite affinity or activity. 

Because many enzymes do not operate as a monomer, the abundance of each catalytically-active complex that catalyzes a given reaction should be included as a separate distinct ``enzyme'' in SIMMER. While we have an incomplete understanding of how complex abundance and activity relates to the abundance of individual components and, in general monomers form stable complexes, while a stoichiometric misbalance between components results in the excess being degraded \textcolor{red}{cite}. In such simple examples, the abundance of a complex would reflect the abundance of each component monomer.  As the genome scale model \textcolor{red}{model} used contains annotated complex membership, to date complex abundance has been estimated using the precision-weighted mean over all complex components. In some cases such a simplified view of complex formation may be inaccurate, however more sophisticated models of complex formation which estimate complex abundance from proteomics data could easily be incorporated into SIMMER.
 
When multiple complexes or isoenzymes can catalyze a single reaction, we are left with two options; either we can assume that each set of enzymes is kinetically similar or different. If we assume that all pools of enzymes interact with metabolites in a similar manner (shared $k_{d}$ and possible $k_{i}$/$k_{a}$) then the occupancy of each enzyme will be the same ($\sfrac{\nu}{\nu_{max}}$) while $k_{cat}$ will vary to reflect that measured enzyme concentrations are relative and sizes of enzyme pools may differ.  This is the standard approach currently used in SIMMER as this minimizes the total number of fitted kinetic parameters, and in most cases a single complex likely carries the bulk of flux. If we chose to relax this assumption, SIMMER is already  compatible with reaction forms where one or more kinetic parameters (besides $k_{eq}$) differ between isoenzymes or complexes. Treating simple Michaelis-Menten kinetics catalyzed by two isoenzyme (E$_{a}$ and E$_{b}$) as an example, reaction flux would follow \hyperref[Eq:isoenzymeMM]{Equation \ref{Eq:isoenzymeMM}}. DAHP synthase (Aro3/Aro4), the first step of aromatic amino synthesis, is one clear example in yeast where such isoenzyme-specific regulation is likely relevant since Aro3p is specifically inhibited by phenylalanine, while Aro4p is specifically inhibited by tyrosine \textcolor{red}{cite}. The kinetics of this reaction were specifically tested for isoenzyme-specific regulation, but the support for this regulation was quantitatively weaker than predicted regulation by phenylpyruvate (a phenylalanine precursor) \textcolor{red}{Section}. 

\begin{equation}
\nu = k_{cat}^{E_{a}}\left[E_{a}\right]\frac{\left[S\right]}{\left[S\right] + k_{m}^{E_{a}}} + k_{cat}^{E_{b}}\left[E_{b}\right]\frac{\left[S\right]}{\left[S\right] + k_{m}^{E_{b}}}\label{Eq:isoenzymeMM}
\end{equation}

Pools of enzymes may differ kinetically because they are composed of different proteins, but the kinetics of otherwise identical enzymes or complexes may also differ if they possess different post-translational modifications.  Post-translational modifications (particularly phosphorylation) are rampant in yeast \cite{Fiedler:2009hx}; their effects on metabolism, however, are largely unknown aside from a few phosphorylation events with an implicated kinetic role, particularly in the TCA cycle \cite{Schulz:2014eo}. While post-translational modifications have not been characterized under the measured conditions, SIMMER could be easily adapted to suggest the physiological role of post-translational modifications in a similar manner to testing allostery. To illustrate how such inference could be conducted, we can consider whether assessing the effect of a single phosphorylation event with a putative effect on enzyme activity explains measured flux significantly better than a model without regulation. If we knew the relative abundance of both the modified protein, $\left[E_{p}\right]$, and unmodified protein $\left[E_{u}\right]$, then reaction kinetics may be described by \hyperref[Eq:ptmMM1]{Equation \ref{Eq:ptmMM1}}, which fits one more parameter than the minimal model.

\begin{equation}
\nu = k_{cat}^{p}\left[E_{p}\right]\frac{\left[S\right]}{\left[S\right] + k_{m}} + k_{cat}^{u}\left[E_{u}\right]\frac{\left[S\right]}{\left[S\right] + k_{m}}\label{Eq:ptmMM1}
\end{equation}

In general, we will not have measured both the modified and unmodified protein, but rather will have measured the abundance of a covalently-modified peptide, $\left[E^{p}_{c}\right]$, and the relative abundance of the whole protein ($\left[E^{p}_{t}\right]$: unmodified plus modified). With this input data, we can not know whether the maximum fraction of covalent modified peptides is 2\% or 50\% of the total protein. To deal with this unknown, the maximum labelling fraction, $\phi_{max} \in [0, 1]$ must be estimated concurrently with kinetic parameters that post-translational modifications are posited to effect (\hyperref[Eq:ptmMM2]{Equation \ref{Eq:ptmMM2}}).

\begin{align}
c^{*} &= \underset{c\in C}{argmax} \sfrac{\left[E^{p}_{c}\right]}{\left[E^{t}_{c}\right]}\notag\\
\phi_{c} &= \phi_{max}\left(\frac{\sfrac{\left[E^{p}_{c}\right]}{\left[E^{t}_{c}\right]}}{\sfrac{\left[E^{p}_{c^*}\right]}{\left[E^{t}_{c^*}\right]}}\right)\notag\\
\nu_c &= k_{cat}^{p}\left[E_c^{p}\right]\frac{\left[S_c\right]}{\left[S_c\right] + k_{m}} + k_{cat}^{t}(1-\phi_c)E_{c}^{t}\frac{\left[S_c\right]}{\left[S_c\right] + k_{m}}\notag\\
 &= \left(k_{cat}^{p}\left[E_c^{p}\right] + k_{cat}^{t}(1-\phi_c)E_{c}^{t}\right) \frac{\left[S_c\right]}{\left[S_c\right] + k_{m}}\label{Eq:ptmMM2}
\end{align}

\subsubsection{Is the ``well-mixed'' assumption appropriate?}

Biochemistry can be greatly simplified when we assume that processes are spatially uniform and thus ``well-mixed''. Assuming that all cells within a culture are effectively equivalent is often an obligate requirement to allow researchers to collect enough physical material to quantify biological species. Extensive culture-wide information can then be abstracted back the content of a single cell. The use of many cells decreases the noise inherent in individual cells, but removes information about how cells meaningfully differ. 

In extreme cases, culture-wide behavior may not reflect the behavior of individual cells. As an example, consider a metabolic reaction: A + B $\rightarrow$ C. If the metabolite A is in present in half of the culture, and B is present in the complementary half, then in neither case will C be produced. However, applying the culture-wide well-mixed assumption interaction between A and B would be possible, allowing the synthesis of C to proceed. An absence of \textit{in vivo} colocalization can lead to the same fallacy when either cellular compartmentalization is ignored, or when enzymes \&/or metabolites are erroneously assumed to be evenly distributed within a compartment.

In some cases the assumption of cell-level similarity is likely appropriate at least to a first approximation. The best-case scenario for such an assumption to hold is likely during chemostat growth, were many generations of similar environmental conditions minimize how environmental heterogeneity and historical contingency affect growth. The culture-wide reproducibility of chemostats suggests the biological variance between individual cells is minimal, while it is also routine to verify the population is morphologically similar, at least in terms of cellular volume, during chemostat culture. While chemostat growth may be a best-case scenario for minimizing cell-to-cell variation, both random \cite{BarEven:2006dz, Kaern:2005gr} and non-random variation at the single cell level may be important at the population-level. The clearest example of a process which varies between cells is the temporal evolution of the mitotic cell-cycle whose sinusoidal oscillations could not be summarized by a bulk-sum analysis \cite{Hartwell:1974uy, Spellman:1998wj}. Such temporally-variably processes are not readily observed unless either cellular population are coordinated\cite{Hartwell:1974uy, Tu:2006cl}, or single cells can be tracked through morphology \cite{Herskowitz:1988ut}, microscopy \cite{Venturelli:2015ec} or single-cell sequencing \cite{Patel:2014dt}. Each of these latter techniques has revealed that otherwise similar cellular populations are frequently composed of distinct clusters of cells, but scope of such heterogeneity remains unclear. If single-cell concentrations of reaction species were available, SIMMER could be applied to model flux heterogeneity, however gaining the necessary input data is not currently possible. While advances in microscopy and automation are enabling the simultaneous tracking of larger numbers of proteins \cite{Ghaemmaghami:2003ds, Dubuis:2013cw}, understanding of single cell metabolic heterogeneity is currently limited by inadequate methods to detect metabolite levels of single cells \cite{Zenobi:2013il}.




culture-wide - cell to cell variability
cell-wide - localization result in constants which are off by a constant, more problematic if either the volume of a compartment changes or the relative distribution of metabolites across compartments is altered.
compartment-wide - channeling, colocalization




addressing the metabolism of yeast, one of the most complicated properties of a cell, I was able to reduce a non-linear system of differential equations, to a set of possible reaction-level models which could be efficiently solved piece-meal. The power of this approach is currently limited by an incomplete characterization of relevant species which prevents an exhaustive analysis of all relevant reactions and all possible models of their kinetics. Despite these current limitations, the approach of SIMMER will remain computationally tractable at scale, and when additional conditions are assessed, this method could practically quantitatively differentiate a true kinetic model to the extent that alternative models can be statistically distinguished by virtue of their variation. While inference in SIMMER proceeds in a reaction-by-reaction manner, inferred kinetic models are sufficient to describe how components interact and ultimately establish pathway or metabolism-level behavior. As such, with the appropriate addition of further experimental data, SIMMER will be an ideal platform for predicting how pathway fluxes change in response to nutrient inputs or alterations of protein expression.

- Assumptions for simplicity
-- Random order

- Metabolomic incompleteness


- Protein activity not reflecting MS data
-- Protein-protein interactions
-- Post-translational regulation
-- localization

- Well mixed assumption
-- variable compartmentalization
-- protein localization / channeling
-- cell-to-cell variability 

- Incompleteness
%-- application to an incomplete # of reactions
-- missing metabolites 

Application to new systems
-- undercharacterized organisms
-- mammalian systems





Using these simple models we determined that for 44 of 71 reactions simple Michaelis-Menten kinetics including at most one regulators could accurately relate metabolite and enzyme concentrations to reaction flux. This is impressive because mechanistic modeling of the kinetics of well-studied reactions rarely reduces to Michaelis-Menten kinetics, and often many regulators are thought to collectively regulate activity \cite{Hill:1977vm}.  Under assay conditions, significant deviations from Michaelis-Menten kinetics may exist and the activity of regulators can be demonstrated; for the purpose of understanding flux control, a reaction form does not need to be principled mechanistically, but rather must only account for an approximately correct relationship between species and resulting flux across the physiological conditions investigated \cite{Fell:1997wg}. Across the diverse conditions that we studied, Michaelis-Menten kinetics is generally appropriate and a single regulator usually dominates even for regulatory hubs such as phosphofructokinase, pyurvate kinase and DAHP synthase.  While our approach would be a powerful way to test the physiological significance of deviations from Michaelis-Menten kinetics or to assess combinatorial regulation, as we demonstrated for the isoenzyme-specific regulation of DAHP synthase, in this manuscript, we have restricted our focus to simpler cases in order to make our inference more systematic.  

% Our approach allows the support for any type of kinetic form to be evaluated.  We focused on generalized Michaelis-Menten kinetics as this provides a well-motivated relationship between substrates, products and enzymes that can be flexibly applied to any reaction based on reaction stoichiometry.  As these simple models are frequently sufficient to explain variable flux at physiological substrate and product concentrations, their simplifying assumptions are likely generally appropriate.  Cases where these assumptions fail and kinetics would be better explained using for instance coopertive binding of substrates or mechanistic kinetic model be determined with this approach.  Because of the added complexity of these models, and consequent increased challenge of accurate parameter estimation and model discrimination, we will defer this investigation. 



Our approach implicitly determines whether reaction species' concentrations can collectively explain measured flux; by interpreting how this alignment has occurred, we can determine how flux changes in response to individual reaction species.  Reversible reactions are near thermodynamic equilibrium, so regulation by allostery (or change in enzyme concentrations) will not change net flux, but rather merely redistribute pathway metabolites.  In line with the passive nature of reversible reactions, we predict that their fluxes are solely driven by the changing concentrations of substrates and products.  Irreversible reactions are points where metabolism is controlled; changes in substrate, enzyme and regulator concentrations will alter flux, but not all irreversible reactions are equally useful for control.  If a reaction's substrate does not change in concentration when the enzyme concentration increases, net flux will increase.  If, by contrast, this substrate decreases as enzyme concentration increases, these effects will frequently cancel.  To differentiate between these cases rigorously, we could generate full pathway models of metabolic control, although by observing how yeast natively alters reaction flux we can still make some conclusions.  If a reaction is strongly regulated through either changes in enzyme concentration or large variable effects of a regulator, then the metabolic leverage provided by this specie likely has some meaningful role.  For glycolysis, we find that pathway analysis using metabolic control analysis and analysis using metabolic leverage both point to phosphofructokinase and pyruvate kinase as the main controlling steps.

In this study, we provide the first indication of how substrates, enzymes and regulators collectively change flux and assess the relative influence of each class.  While reactions are diverse, changes in substrate concentrations are important for the majority of reactions; changes in enzyme and regulator concentrations have a smaller role.  End-product inhibition is important for many canonical biosynthetic pathways, but these feedbacks are by no means the primary determinant of pathway flux.  In most cases, substrates and enzymes are approximately balanced to provide appropriate flux, and to the extent that they are misbalanced, altered concentrations of the end-product fulfill a secondary role of fine-tuning flux to an appropriate rate.  Although our analysis proceeds on a reaction-by-reaction basis, we are largely able to predict the reactions that are important regulated steps in metabolism. For reactions that are strongly regulated, we can predict the extent to which this control is affected at the level of protein abundance versus small molecule regulation.  This a valuable tool for metabolic engineers because we are able to determine when the metabolome appears to diminish changes in enzyme concentrations and identify regulation that should be considered when attempting to change pathway flux.  

While our data is the most comprehensive enzyme-metabolite-flux dataset to date, it is still not a complete collection of all relevant metabolic information. Techniques to quantify these species continue to advance at a meteoric pace; more accurate and thorough analysis will partly rely upon these technological improvements and may additionally require more focused analytical methods. For 44 reactions, we determined approximately correct kinetics, while for 27 reactions we tested and rejected all simple models of reaction kinetics that included only reaction species and proposed regulators.  Although in some cases, our kinetic predictions may be inaccurate because we did not assess deviations from Michaelis-Menten kinetics or combinatorial regulation, in most cases, reactions are probably poorly predicted because we have not considered additional regulation or heterogeneity.  Post-translational regulation, variable sub-cellular localization, metabolic channeling or large cell-to-cell variability may play a role in many reactions, however, it is difficult to say which of these phenomena are relevant.  Post-translational modifications (particularly phosphorylation) are rampant in yeast \cite{Fiedler:2009hx}; their effects on metabolism, however, are largely unknown aside from a few phosphorylation events with an implicated kinetic role, particularly in the TCA cycle \cite{Schulz:2014eo}. If metabolites are differentially partitioned across conditions (between the cytosol, mitochondria and vacuole in particular), relative concentrations in compartments will not reflect that of whole-cells \cite{Kitamoto:1988wc}. Differential partitioning of phenylalanine could explain why phenylpyruvate was predicted as an inhibitor of DAHP synthase rather than phenylalanine because phenylpyruvate is the source of cytosolic phenylalanine synthesis.  In addition to the vacuole, concentrations of mitochondrial metabolites are likely unreliable. This is  because relative mitochondrial volume may vary across conditions, as suggested by the strong covariation of mitochondrial enzymes (Supplemental Figure 2) and because many mitochondrial metabolites are also present in the cytosol but do not readily mix.  While methods to reliably determine the concentrations of sub-cellular active metabolites in metabolically active compartments are lacking, such methods may be necessary to kinetically model the yeast TCA cycle \cite{Schulz:2014eo}.  Cell-to-cell variability may also be relevant in some cases; while we have avoided conditions where yeast do not undergo synchronous metabolic cycling, it is unclear how much metabolism of individual cells will vary over time and from cell to cell. While these caveats represent the limitations of this study, they are limitations due to the scope of data that we could collect, and such data could be readily assimilated into the framework that we have constructed.  

Because we have characterized a reproducible set of conditions that contain much of the meaningful metabolic variation in yeast, we hope that this study will serve as an important scaffold for future work to build upon, whether by incorporating additional experimental information on existing conditions or by expanding the breadth of the study by characterizing additional conditions. We can only identify important regulation that impacts the conditions that we have studied. By expanding the number of conditions that are investigated, we can unmask regulation that is important in these conditions, and our approach can identify this regulation by virtue of the kinetic importance it gains in these new conditions. This will also serve to refine the accuracy of parameter estimates from reaction forms and allow competing regulation to be more readily distinguished.



\subsection{Towards a Comprehensive Model of Protein Expression}

By analyzing systematic deviations between proteins and transcripts during Chapter \ref{ch:pt_compare} I implicated post-transcriptional regulation as providing a highly relevant layer of protein-expression control. This analysis was not ideal due to differences in experimental design between the two compared datasets. By measuring proteins and transcripts in the same culture and ideally accounting for biological variance through replication, a more faithful comparison between these species could be conducted.  Such an analysis could also disambiguate the role of regulation of translational efficiency as opposed to protein degradation through experimental determination of one (or both) of these processes. Measuring translational efficiency through ribosome profiling is becoming more routine   . Estimation of protein degradation rate   .

By establishing a quantitative central dogma of molecular biology, through understanding how protein levels are established, predicted changes in expression in response to environmental or genetic variation could be assessed. 
Establishing a quantitative central dogma

Metabolism x expression models


